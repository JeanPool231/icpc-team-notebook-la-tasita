%%%%%%%%%%%%%%%%%%%%%%%%%%%%%%%%%%%%%%%%%%%%%%%%%%%%%%%%%%%%
% NOTEBOOK ICPC - LA TASITA (VERSIÓN MEJORADA)
%%%%%%%%%%%%%%%%%%%%%%%%%%%%%%%%%%%%%%%%%%%%%%%%%%%%%%%%%%%%

% --- CLASE DE DOCUMENTO ---
\documentclass[12pt, a4paper]{report}

% --- PAQUETES ESENCIALES ---
\usepackage[utf8]{inputenc}    % Acentos y ñ
\usepackage[T1]{fontenc}       % Codificación de fuentes
\usepackage[spanish]{babel}    % Idioma español (para "ÍndGice", "Capítulo")
\usepackage{geometry}          % Márgenes personalizados
\geometry{left=2.5cm, right=2.5cm, top=2.5cm, bottom=2.5cm}

% --- PAQUETES PARA CÓDIGO ---
\usepackage{listings}          % Para insertar código
\usepackage{xcolor}            % Para colores en el código

% --- PAQUETES ADICIONALES ---
\usepackage{amsmath, amssymb}  % Para fórmulas matemáticas
\usepackage{graphicx}          % Para imágenes
\usepackage{hyperref}          % Para links (útil en el índice)

% --- CONFIGURACIÓN DE LISTINGS (CÓDIGO) ---
\definecolor{codegreen}{rgb}{0,0.6,0}
\definecolor{codegray}{rgb}{0.5,0.5,0.5}
\definecolor{codepurple}{rgb}{0.58,0,0.82}
\definecolor{backcolour}{rgb}{0.97, 0.97, 0.97}

\lstdefinestyle{mystyle}{
    language=C++,
    backgroundcolor=\color{backcolour},   
    commentstyle=\color{codegreen}\itshape,
    keywordstyle=\color{blue}\bfseries,
    numberstyle=\tiny\color{codegray},
    stringstyle=\color{codepurple},
    basicstyle=\ttfamily\footnotesize,
    breakatwhitespace=false,         
    breaklines=true,                 
    captionpos=b,                    
    keepspaces=true,                 
    numbers=left,                    
    numbersep=5pt,                  
    showspaces=false,                
    showstringspaces=false,
    showtabs=false,                  
    tabsize=2
}
\lstset{style=mystyle} % Aplicar este estilo por defecto

%%%%%%%%%%%%%%%%%%%%%%%%%%%%%%%%%%%%%%%%%%%%%%%%%%%%%%%%%%%%
% --- MODIFICACIONES DEL PREÁMBULO ---
%%%%%%%%%%%%%%%%%%%%%%%%%%%%%%%%%%%%%%%%%%%%%%%%%%%%%%%%%%%%

% --- 1. FORZAR NUEVA PÁGINA POR SECCIÓN ---
% Redefine \section para que siempre inicie en una página nueva.
\let\oldsection\section
\renewcommand{\section}{\clearpage\oldsection}

% --- 2. ARREGLO PARA USAR \part PERO NUMERAR SECCIONES 1.1, 2.1 ---
% Asegura que \section y \subsection se numeren y aparezcan en el índice
\setcounter{secnumdepth}{2} 
\setcounter{tocdepth}{2}    


%%%%%%%%%%%%%%%%%%%%%%%%%%%%%%%%%%%%%%%%%%%%%%%%%%%%%%%%%%%%
% INICIO DEL DOCUMENTO
%%%%%%%%%%%%%%%%%%%%%%%%%%%%%%%%%%%%%%%%%%%%%%%%%%%%%%%%%%%%
\begin{document}

% --- PORTADA ---
\begin{titlepage}
    \centering
    \vspace*{\fill}
    
    {\Huge \textbf{La tasita}} % Título principal
    
    \vfill
    
    {\Large ICPC Notebook} % Subtítulo
    
    \vfill
    
    {\large Team: Turistas } % Nombre del equipo
    
    \vfill
\end{titlepage}

% --- ÍNDICE (TABLA DE CONTENIDO) ---
\tableofcontents % Genera el índice automáticamente
\newpage

%%%%%%%%%%%%%%%%%%%%%%%%%%%%%%%%%%%%%%%%%%%%%%%%%%%%%%%%%%%%
% INICIO DEL CONTENIDO DEL NOTEBOOK
%
% ESTRUCTURA DE USO:
% 1. Usa \part{...} para el título principal del tema (Parte I, II...)
% 2. Usa \chapter{} INMEDIATAMENTE DESPUÉS para resetear el contador.
% 3. Usa \section{...} para tu algoritmo (se numerará 1.1, 1.2...)
% 4. Usa \lstinputlisting{...} para jalar el código desde un archivo.
%%%%%%%%%%%%%%%%%%%%%%%%%%%%%%%%%%%%%%%%%%%%%%%%%%%%%%%%%%%%

\part{Dynamic Programming}
% --- Reseteo de contadores
\stepcounter{chapter} % Avanza el contador principal (invisible)
\setcounter{section}{0} % Resetea el contador de sección a 0

\section{Longest Common Subsequence} % \clearpage se activa aquí
\lstinputlisting{DP/LCS.cpp}

\section{Longest Increasing Subsequence} 
\lstinputlisting{DP/LIS.cpp}

\section{Longest Increasing Subsequence Fast}
\lstinputlisting{DP/LISFAST.cpp}

\section{Maximum Subarray}
\lstinputlisting{DP/MaximumSubarray.cpp}

\section{Knapsack}
\lstinputlisting{DP/Knapsack.cpp}

\section{Knapsack Binary}
\lstinputlisting{DP/KnapsackBinary.cpp}

\section{Knapsack Re}
\lstinputlisting{DP/KnapsackRe.cpp}

\section{Knapsack Subset Sum}
\lstinputlisting{DP/KnapsackSubsetSum.cpp}

\part{Data Structures}
\stepcounter{chapter}
\setcounter{section}{0}

\section{Fenwick Tree}
\lstinputlisting{data-structures/fenwick-tree.cpp}

\section{Fenwick Tree 2d}
\lstinputlisting{data-structures/fenwick-tree-2d.cpp}
\section{Treap}
\lstinputlisting{data-structures/treap.cpp}
\section{Implicit Treap}
\lstinputlisting{data-structures/implicit-treap.cpp}
\section{Ordered Set}
\lstinputlisting{data-structures/ordered_set.cpp}
\section{Segment Tree}
\lstinputlisting{data-structures/segment-tree.cpp}
\section{Segment Tree Summary}
\lstinputlisting{data-structures/segment-tree-summary.cpp}
\section{Persistent Segment Tree}
\lstinputlisting{data-structures/persistent-segment-tree.cpp}
\section{Sparse Table}
\lstinputlisting{data-structures/sparse-table.cpp}

\part{Geometry}
\stepcounter{chapter}
\setcounter{section}{0}

\section{Convex Hull}
\lstinputlisting{geometry/convex-hull.cpp}

\part{Graphs}
\stepcounter{chapter}
\setcounter{section}{0}
\section{2 Sat}
\lstinputlisting{graphs/2-sat.cpp}
\section{Bellman Ford}
\lstinputlisting{graphs/bellman-ford.cpp}
\section{Dijkstra}
\lstinputlisting{graphs/dijkstra.cpp}
\section{Floyd Warshall}
\lstinputlisting{graphs/floyd-warshall.cpp}
\section{Ford Fulkerson}
\lstinputlisting{graphs/ford-fulkerson.cpp}
\section{Functional Graph}
\lstinputlisting{graphs/functional-graph.cpp}
\section{Hungarian}
\lstinputlisting{graphs/hungarian.cpp}
\section{Kosaraju}
\lstinputlisting{graphs/kosaraju-scc.cpp}
\section{Kruskal}
\lstinputlisting{graphs/kruskal.cpp}
\section{Heavy Light Descomposition}
\lstinputlisting{graphs/HLD.cpp}
\section{LCA}
\lstinputlisting{graphs/lca.cpp}
\section{Topological Sort}
\lstinputlisting{graphs/topological-sort.cpp}
\section{Centroid Descomposition}
\lstinputlisting{graphs/CD.cpp}



\part{Math}
\stepcounter{chapter}
\setcounter{section}{0}
\section{FFT}
\lstinputlisting{math/FFT.cpp}
\section{NTT}
\lstinputlisting{math/NTT.cpp}
\section{Binary Exponentiation}
\lstinputlisting{math/binary-exponentiation.cpp}
\section{Matrix Exponentiation}
\lstinputlisting{math/matrix-exponentiation.cpp}

\part{Strings}
\stepcounter{chapter}
\setcounter{section}{0}
\section{KMP Automaton}
\lstinputlisting{strings/kmp-automaton.cpp}
\section{Hashing}
\lstinputlisting{strings/string-hashing.cpp}
\end{document}
